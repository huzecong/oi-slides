%!TEX program = xelatex
\documentclass[11pt,dvipsnames,table]{beamer}
\usepackage[no-math,cm-default]{fontspec}
\usepackage[indentfirst]{xeCJK}
\usepackage{amsmath}
\usepackage{amsthm}
\usepackage{amssymb}
\usepackage{verbatim}
\usepackage{indentfirst}
\usepackage{syntonly}
\usepackage{beamerthemesplit}
\usepackage{ulem}
\usepackage{listings}
\usepackage{etoolbox}

\usetheme{Berlin}
\usecolortheme{beaver}
\usefonttheme[onlymath]{serif}
\CJKsetecglue{}
\setbeamerfont{section in toc}{size=\fontsize{10pt}{\baselineskip}}

%\setbeamerfont{normal text}{size=9pt}

\setCJKmainfont[BoldFont={SimHei},ItalicFont={KaiTi}]{SimSun}
\setmonofont[Scale=1]{Courier New}

\newcommand{\hlink}[1]{
	\footnote{\fontsize{6pt}{\baselineskip}\href{#1}{\textsl{\underline{#1}}}}
}
\newcommand{\image}[4]
{\begin{figure}[h]
	\centering
	\includegraphics[width=#3 \textwidth]{image/#2}
	\caption{#4}\label{#1}
\end{figure}}
\newcommand{\graph}[2]
{\begin{figure}[h]
	\centering
	\includegraphics[width=#2 \textwidth]{image/#1}
\end{figure}}

\lstset{language=C++,
	extendedchars=false,
	basicstyle=\ttfamily\footnotesize,
	keywordstyle=\bfseries\color{blue},
	identifierstyle=\color{blue!60!black},
	commentstyle=\itshape\color{gray},
	escapeinside=`'}

\setlength{\parindent}{2em}
\setlength{\baselineskip}{1.3\baselineskip}

\setbeamercolor{math text}{fg=black}
\setbeamertemplate{qed symbol}{$\square$}
\setbeamerfont{headline}{size=\fontsize{7.5pt}{\baselineskip}}
\setbeamerfont{footline}{size=\fontsize{7.5pt}{\baselineskip}}
\setbeamertemplate{theorems}[numbered]
\renewcommand{\thetheorem}{\arabic{subsubsection}.\arabic{theorem}}
\renewcommand{\thelemma}{\arabic{subsubsection}.\arabic{lemma}}
\newenvironment{qedframe}{%
	\begin{frame}[environment=qedqedframe]%
	}{%
	\qed
	\end{frame}%
}
\renewcommand{\appendixname}{结语}

\makeatletter
\patchcmd{\beamer@sectionintoc}{\vskip1.5em}{\vskip1em}{}{}
\makeatother

\begin{document}

\title[网络流]{\fontsize{24pt}{\baselineskip}网络流}
\subtitle[模型与例题]{\fontsize{16pt}{\baselineskip}模型与例题\\ 进阶篇}
\author{清华大学计算机系~~胡泽聪}
\institute[清华大学计算机系~~胡泽聪]{}
\date{}

\maketitle

{\fontsize{9pt}{\baselineskip}

\section{前言}
\subsection{}
\begin{frame}
	无需多言。
\end{frame}

\section[Model VII]{限制距离}
\subsection{}
\begin{frame}
	\frametitle{基础概念}
	这类模型是一类网络流最小割模型,用于描述每个点有若干种不同选择,标号为连续的整数,选择有各自的代价。同时有若干限制条件,为$x_i-x_j\leq d$的形式,其中$x_i$和$x_j$为$i$和$j$的选择的标号。
\end{frame}

\subsection{HNOI2013 (BZOJ3144)}
\begin{frame}
	\frametitle{切糕\hlink{http://www.lydsy.com/JudgeOnline/problem.php?id=3144} - 题意}
	$p\times q$的网格,每个位置都有$r$个取值的选择,每个选择有各自的代价。要求四连通相邻的两个位置的值相差不超过$d$。问最小代价和。
	
	$p,q,r\leq 40$。
\end{frame}
\begin{frame}
	\frametitle{切糕 - 模型}
	对每个位置的每个选择设点,形成一条链,相邻位置的边的流量为选择的代价。\pause
	
	由于每个位置只能选择一个取值,因此自然想到最小割。现在加入距离限制。限制可以描述为$x_i\geq x_j-d$,其中$i$和$j$相邻。网络流中的限制通常用无穷大的边来描述。\pause
	
	因此对于相邻的$(x,y)$和$(x',y')$以及$d\leq k\leq r$,连边$(x,y,k)\rightarrow(x',y',k-d):\infty$。
	
	这个图的最小割就是答案。
\end{frame}

\subsection{TopCoder SRM590 D1L3}
\begin{frame}
	\frametitle{FoxAndCity\hlink{http://community.topcoder.com/stat?c=problem\_statement\&pm=12727} - 题意}
	给定$n$个点的无向图,边权均为$1$。每个点有一个属性$w_i$。现在可以在图中任意加边,记加边后每个点到1号点的距离为$d_i$,最小化$\sum(w_i-d_i)^2$。
\end{frame}
\begin{frame}
	\frametitle{FoxAndCity - 模型}
	我们考察一下最终的距离应当满足怎样的性质。
	\begin{itemize}
		\item[性质1] $d_1=0$。\pause
		\item[性质2] 对于已经存在的边$i\leftrightarrow j$,必有$|d_i-d_j|\leq 1$。\pause
	\end{itemize}
	
	除此之外呢?\pause
	
	事实上,只要满足这两个条件,就能构造出符合条件的图。\pause
	\vskip 1em
	此时相当于每个点有$n-1$个选择,原本有边相连的两点$i$和$j$应满足$d_i-d_j\leq 1$以及$d_j-d_i\leq 1$。变成了上面一题的模型。
\end{frame}

\subsection{CTSC2009}
\begin{frame}
	\frametitle{移民站选址\footnote{\fontsize{6pt}{\baselineskip}提交答案题} - 题意}
	已有$n$个移民站,第$i$个坐标为$(u_i,v_i)$。还需建立$m$个新的移民站,记坐标为$(x_i,y_i)$。移民站之间需要传输数据,第$i$个旧站和第$j$个新站之间要传$a_{ij}$的数据,第$i$个新站和第$j$个新站之间要传$b_{ij}$的数据。传输代价是两个站之间传输的数据量乘上两个站之间的曼哈顿距离。求最小代价和。
	
	虽说是提交答案题,但可以当成普通题来做。
\end{frame}
\begin{frame}
	\frametitle{移民站选址 - 模型}
	我们需要先得出一些结论:
	\begin{itemize}
		\item[结论1] 横纵坐标无关,可以分开当成一维问题做。\pause
		\item[结论2] 一定存在一个最优方案,使得每个新站都与某个旧站重合。
	\end{itemize}
\end{frame}
\begin{frame}
	\frametitle{移民站选址 - 模型}
	那么问题可以变成,$m$个位置,每个有$n$种选择,代价即为其与旧站的传输代价和。不同位置间的选择也会带来代价。\pause
	
	我们这样连边:对于任意的新站$i$和$j$以及任意的$1<k\leq n$,连边$(i,k)\leftrightarrow (j,k):b_{ij}$。不难发现,如果$i$的选择是$p_i$,$j$的选择是$p_j$,则还会有$|p_i-p_j|$条这样的边被割掉,即为应有的传输代价。
	
	最后的答案就是最小割。
\end{frame}

\appendix
\section{}
\begin{frame}
	\frametitle{Fin.}
	\centering
	\LARGE
	谢谢大家!欢迎课后交流。
\end{frame}
}

\end{document}
