%!TEX program = xelatex
\documentclass[11pt,dvipsnames,table]{beamer}
\usepackage[no-math,cm-default]{fontspec}
\usepackage[indentfirst]{xeCJK}
\usepackage{amsmath}
\usepackage{amsthm}
\usepackage{amssymb}
\usepackage{verbatim}
\usepackage{indentfirst}
\usepackage{syntonly}
\usepackage{beamerthemesplit}
\usepackage{ulem}
\usepackage{listings}
\usepackage{etoolbox}

\usetheme{Berlin}
\usecolortheme{beaver}
\usefonttheme[onlymath]{serif}
\CJKsetecglue{}
\setbeamerfont{section in toc}{size=\fontsize{10pt}{\baselineskip}}

%\setbeamerfont{normal text}{size=9pt}

\setCJKmainfont[BoldFont={SimHei},ItalicFont={KaiTi}]{SimSun}
\setmonofont[Scale=1]{Courier New}

\newcommand{\hlink}[1]{
	\footnote{\fontsize{6pt}{\baselineskip}\href{#1}{\textsl{\underline{#1}}}}
}
\newcommand{\image}[4]
{\begin{figure}[h]
	\centering
	\includegraphics[width=#3 \textwidth]{image/#2}
	\caption{#4}\label{#1}
\end{figure}}
\newcommand{\graph}[2]
{\begin{figure}[h]
	\centering
	\includegraphics[width=#2 \textwidth]{image/#1}
\end{figure}}

\lstset{language=C++,
	extendedchars=false,
	basicstyle=\ttfamily\footnotesize,
	keywordstyle=\bfseries\color{blue},
	identifierstyle=\color{blue!60!black},
	commentstyle=\itshape\color{gray},
	escapeinside=`'}

\setlength{\parindent}{2em}
\setlength{\baselineskip}{1.3\baselineskip}

\setbeamercolor{math text}{fg=black}
\setbeamertemplate{qed symbol}{$\square$}
\setbeamerfont{headline}{size=\fontsize{7.5pt}{\baselineskip}}
\setbeamerfont{footline}{size=\fontsize{7.5pt}{\baselineskip}}
\setbeamertemplate{theorems}[numbered]
\renewcommand{\thetheorem}{\arabic{subsubsection}.\arabic{theorem}}
\renewcommand{\thelemma}{\arabic{subsubsection}.\arabic{lemma}}
\newenvironment{qedframe}{%
	\begin{frame}[environment=qedqedframe]%
	}{%
	\qed
	\end{frame}%
}
\renewcommand{\appendixname}{结语}

\makeatletter
\patchcmd{\beamer@sectionintoc}{\vskip1.5em}{\vskip1em}{}{}
\makeatother

\begin{document}

\title[网络流]{\fontsize{24pt}{\baselineskip}网络流}
\subtitle[模型与例题]{\fontsize{16pt}{\baselineskip}模型与例题\\ 进阶篇}
\author{清华大学计算机系~~胡泽聪}
\institute[清华大学计算机系~~胡泽聪]{}
\date{}

\maketitle

{\fontsize{9pt}{\baselineskip}

\section{前言}
\subsection{}
\begin{frame}
	无需多言。
\end{frame}

\section[Model VIII]{区间选择}
\subsection{}
\begin{frame}
	\frametitle{区间选择}
	对于长度为$n$的数组,我们设立$n+1$个点构成一条链,相邻的点之间连边,费用为对应数组中元素的值。再从$S$向前$n$个点,从后$n$个点向$T$连边,一条增广路就对应一个区间。
	
	非常简洁而且直观的模型。
	
	然而,简洁却不简单。
\end{frame}
\subsection{TCO09 Championship Round D1L2}
\begin{frame}
	\frametitle{ArrayTransformations\hlink{http://community.topcoder.com/stat?c=problem\_statement\&pm=10428} - 题意}
	给定序列$a$,记$a$的元素个数为$n$。
	
	称一次变换为:选定区间$[i,j]$,满足$1\leq i\leq j\leq n$,对序列$a$在区间中的每个数减1,如果已经为0则不操作。这样一次变换的代价为$j-i+1$。
	
	对于给定序列$a$,最大变换次数$k$和最大代价和$m$,求:在使用不超过$k$次变换,总变换代价不超过$m$的前提下,$a$中最大元素的最小值是多少。
	
	$n\leq 250$。
\end{frame}
\begin{frame}
	\frametitle{ArrayTransformations - 模型}
	首先二分答案,把问题转换成判定性问题。
	
	现在的问题是,能否在限制下使$a$中最大元素不超过一个值$p$。\pause
	
	容易发现,变换的顺序并不影响最后的结果。
	
	对于$a_i$,变换的最小代价应为$\max(a_i-P,0)$。那么我们令$b_i=\max(a_i-P,0)$。
	
	那么限制条件实际上就是要确定不超过$K$条线段,使得数轴上$i$的位置被覆盖了至少$b_i$次,且线段总长不超过$M$。
\end{frame}
\begin{frame}
	\frametitle{ArrayTransformations - 模型}
	据此可以建立下面的网络流模型:
	
	建立标号为1到$n+1$的点,以及$S'$和$T'$节点。
	
	从$S$向$S'$连一条容量为$K$的边、从$T'$向$T$连一条容量为$K$、费用为0的边。
	
	从$i$号点$i+1$号点连一条容量为$\infty$、容量下界为$a_i-P$、费用为1的边。
	
	从$S'$向$1\sim n$号节点、从$2\sim n+1$号节点向$T'$连一条容量为$\infty$、费用为0的边。
	
	一次变换,即一条线段$[i,j]$,在网络中会是$S\rightarrow S'\rightarrow i\rightarrow i+1\rightarrow \ldots\rightarrow j\rightarrow T'\rightarrow T$的一条增广路。
	
	如果最小费用最大可行流得到的费用不超过$M$,那么答案合法。
\end{frame}

\subsection{Codeforces Round \#172 (Div. 1) - D}
\begin{frame}
	\frametitle{k-Maximum Subsequence Sum\hlink{http://codeforces.com/contest/280/problem/D} - 题意}
	给定一个长度为$n$的序列,有$q$次操作。操作有2种:
	\begin{enumerate}
		\item 修改一个元素的值。
		\item 查询一个区间中的最大$k-$子段和。
	\end{enumerate}
	$n,m\leq 10^5$,$k\leq 20$。
\end{frame}
\begin{frame}
	\frametitle{k-Maximum Subsequence Sum - 模型}
	吓到了吧,这题可是大数据结构题,怎么会和网络流有关?
	
	首先单次$O(k^2\log n)$的算法是不难的,但是会超时。我们需要一个$O(k\log n)$的算法。\pause
	\vskip 1em
	我们按照类似上题中的方法建费用流模型。显然最大费用最大流就是答案。但这样反而比暴力还慢。
	
	不妨考察连续最短路算法的执行过程:我们每次是在干什么?
\end{frame}
\begin{frame}
	\frametitle{k-Maximum Subsequence Sum - 模型}
	结合这个模型的性质,我们可以发现我们每次要么选择一个与已有区间完全不相交的区间,要么从已有区间中间挖掉一段。
	
	换句话说就是:每次贪心选择一个最大和子区间,然后取负。
	
	我们还可以发现,每次都会使区间数+1。因此我们可以用线段树维护最大子段和,并支持取负操作,然后每次询问就在线段树上询问$k$次,之后复原即可。修改也很容易实现。
\end{frame}

\appendix
\section{}
\begin{frame}
	\frametitle{Fin.}
	\centering
	\LARGE
	谢谢大家!欢迎课后交流。
\end{frame}
}

\end{document}
