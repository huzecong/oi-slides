%!TEX program = xelatex
\documentclass[9pt,dvipsnames,table]{beamer}
\usepackage[no-math,cm-default]{fontspec}
\usepackage[indentfirst]{xeCJK}
\usepackage{amsmath}
\usepackage{amsthm}
\usepackage{amssymb}
\usepackage{verbatim}
\usepackage{indentfirst}
\usepackage{syntonly}
\usepackage{beamerthemesplit}
\usepackage{euler}
\usepackage{ulem}
\usepackage{listings}
\usepackage{etoolbox}
\usepackage{zhspacing}

\usetheme{Berlin}
\usecolortheme{beaver}
%\usefonttheme[onlymath]{serif}
\usefonttheme{professionalfonts}
\CJKsetecglue{}
\setbeamerfont{section in toc}{size=\fontsize{10pt}{\baselineskip}}

\setCJKmainfont[BoldFont={SimHei},ItalicFont={KaiTi}]{SimSun}
\setmonofont[Scale=1]{Consolas}

\newcommand{\hlink}[1]{
	\footnote{\fontsize{6pt}{\baselineskip}\href{#1}{\textsl{\underline{#1}}}}
}
\newcommand{\graph}[2]
{\begin{figure}[h]
	\centering
	\includegraphics[width=#2 \textwidth]{image/#1}
\end{figure}}

\lstset{language=C++,
	extendedchars=false,
	basicstyle=\ttfamily\footnotesize,
	keywordstyle=\bfseries\color{blue},
	identifierstyle=\color{blue!60!black},
	commentstyle=\itshape\color{gray},
	escapeinside=`'}

\setlength{\baselineskip}{1.3\baselineskip}
\setlength{\parindent}{2em}

\setbeamercolor{math text}{fg=black}
\setbeamertemplate{qed symbol}{ $ \square $ }
\setbeamerfont{headline}{size=\fontsize{7.5pt}{\baselineskip}}
\setbeamerfont{footline}{size=\fontsize{7.5pt}{\baselineskip}}
\setbeamertemplate{theorems}[numbered]
\renewcommand{\thetheorem}{\arabic{subsubsection}.\arabic{theorem}}
\renewcommand{\thelemma}{\arabic{subsubsection}.\arabic{lemma}}
\newenvironment{qedframe}{%
	\begin{frame}[environment=qedqedframe]%
	}{%
	\qed
	\end{frame}%
}
\renewcommand{\appendixname}{结语}

\makeatletter
\patchcmd{\beamer@sectionintoc}{\vskip 1.5em}{\vskip 1em}{}{}
\makeatother

\begin{document}

\title[数据结构及其应用]{\fontsize{24pt}{\baselineskip}\textbf{数据结构及应用}}
\subtitle[]{\fontsize{16pt}{\baselineskip}Data Structures and their Applications}
\author{清华大学计算机系~~胡泽聪}
%\institute[清华大学计算机系~~胡泽聪]{}
\date{}

\maketitle

%\setcounter{tocdepth}{1}
\begin{frame}
	\frametitle{目录}
	\tableofcontents[hideallsubsections]
\end{frame}

\section{前言}
\subsection{}
\begin{qedframe}
	\frametitle{前言}
	什么是数据结构?
	\begin{itemize}
		\item 基础数据结构
		\item 高级数据结构
	\end{itemize}
	
	\vskip 1em
	数据结构在OI中有何应用?
	\vskip 1em
	数据结构在OI中处于怎样的地位?
\end{qedframe}
\begin{qedframe}
	\frametitle{数据结构不止是结构}
	\begin{itemize}
		\item \textbf{数学推导:}如何维护数据……
		\item \textbf{性质利用:}单调性、区间减法……
		\item \textbf{思维方法:}分治、分块、离线……
	\end{itemize}
\end{qedframe}


\section[Part I]{平衡树}
\title[数据结构及其应用]{\fontsize{18pt}{\baselineskip}\textbf{平衡树}}
\subtitle[]{Balanced Trees}
\author[清华大学计算机系~~胡泽聪]{}
\maketitle

\subsection{基础知识}
\begin{qedframe}
	\frametitle{二叉查找树}
	二叉树,节点带权值。
	
	满足左儿子权值小于等于自己,右儿子权值大于等于自己。\pause
	
	形态任意。(有多少种不同形态?)\pause
	
	中序遍历即为权值的有序序列。
\end{qedframe}
\begin{qedframe}
	\frametitle{树的操作}
	\begin{itemize}
		\item 构建;
		\item 插入;
		\item 删除;
		\item 查找元素;
		\item 寻找前驱/后继。
	\end{itemize}
\end{qedframe}
\begin{qedframe}
	\frametitle{树的平衡}
	在插入、删除后,我们无法保证树足够平衡。
	
	最坏情况下(顺序插入),树高可以达到 $ O(n) $ 级别。而查找的复杂度是树高级别的。
	
	我们需要一些方法调整树的高度。
\end{qedframe}
\begin{qedframe}
	\frametitle{树的旋转}
	\begin{columns}
		\begin{column}{0.5\textwidth}
			\graph{Tree_rotation.png}{}
		\end{column}
		\begin{column}{0.5\textwidth}
			执行这种旋转操作后,性质保持。
			
			右旋又叫zig,左旋又叫zag。
			
			可以通过旋转将树变为任意形态。
		\end{column}
	\end{columns}
\end{qedframe}

\subsection{Splay}
\begin{qedframe}
	\frametitle{Splay}
	Splay的思路很简单。
	
	既然形态可以任意变化,我不妨把我要操作的东西挪到根附近,这样插入删除都比较方便。
	
	splay操作就是将点提到根:通过判断与父亲的关系不断zig或zag。
	
	拓展一下,其实可以将点提到任意(满足性质的前提下)确定的位置。
\end{qedframe}
\begin{qedframe}
	\frametitle{Splay的各种操作}
	\begin{itemize}
		\item \textbf{splay:}通过判断与父亲的关系不断zig或zag,直到成为根;
		\item \textbf{insert:}先将前驱提到根,再将后继提到根的右儿子,插入到根右儿子的左儿子处;
		\item \textbf{remove:}与insert相反。
	\end{itemize}
\end{qedframe}
\begin{qedframe}
	\frametitle{单旋与双旋}
	光是这样,Splay是无法保证复杂度的。
	
	一个例子就是顺序插入之后,不断访问第一个和最后一个元素。单次操作 $ O(n) $ 。\pause
	
	Tarjan \& Sleator (1985) 提出了双旋的办法解决这个问题。
	
	方法很简单:splay操作中,不光考虑和父亲的关系,还考虑父亲和祖父的关系。如果都是左儿子或者都是右儿子,那么先旋转父亲,再旋转自己。
	
	直观理解,这样在一条长链的情况下,可以有效减小树高。
	
	通过势能分析可以证明,操作的均摊复杂度为 $ O(\log n) $ 。
\end{qedframe}
\begin{qedframe}
	\frametitle{Splay的区间操作}
	Splay如果光是这样,是没有太多前途的。
	
	由于Splay形态高度自由(与其他平衡树对比),Splay可以完成几乎所有线段树的区间操作。
	\begin{itemize}
		\item \textbf{select:}将左端点的前驱提到根,右端点的后继提到根的右儿子;为了方便操作可以插入起止虚节点;
		\item \textbf{各种线段树操作:}select出来之后对子树操作,维护时稍有区别;
		\item \textbf{区间翻转:}select出来之后打翻转标记,push的时候处理。这个是线段树做不到的。
	\end{itemize}
\end{qedframe}

\subsection{平衡树例题}
\begin{qedframe}
	\frametitle{POJ3580 SuperMemo\hlink{http://poj.org/problem?id=3580}}
	\begin{columns}
		\begin{column}{0.5\textwidth}
			  给定初始长度为 $ n $ 的序列,有 $ q $ 次操作,操作有六种:
			\begin{itemize}
				\item 区间 $ +k $ ;
				\item 区间翻转;
				\item 区间旋转;
				\item 插入单个元素;
				\item 删除单个元素;
				\item 询问区间最小值。
			\end{itemize}
		\end{column} \pause
		\begin{column}{0.5\textwidth}
			  标准题目。
		\end{column}
	\end{columns}
\end{qedframe}
\begin{qedframe}
	\frametitle{NOI2005 维修数列\hlink{http://www.lydsy.com/JudgeOnline/problem.php?id=1500}}
	\begin{columns}
		\begin{column}{0.5\textwidth}
			  给定初始长度为 $ n $ 的序列,有 $ q $ 次操作,操作有六种:
			\begin{itemize}
				\item 插入一段元素;
				\item 删除一段元素;
				\item 区间赋值;
				\item 区间翻转;
				\item 询问区间和;
				\item 询问区间最大子段和。
			\end{itemize}
		\end{column} \pause
		\begin{column}{0.5\textwidth}
			  标准题目2。
			
			  插入一段的时候,先建出一棵完全二叉树。
			
			  最大子段和如何维护?
		\end{column}
	\end{columns}
\end{qedframe}
\begin{qedframe}
	\frametitle{ZOJ2112 Dynamic Rankings\hlink{http://acm.zju.edu.cn/onlinejudge/showProblem.do?problemCode=2112}}
	\begin{columns}
		\begin{column}{0.5\textwidth}
			  给定长度为 $ n $ 的序列,有 $ q $ 次操作,操作有两种:
			\begin{itemize}
				\item 修改单个元素;
				\item 求区间第 $ k $ 大的元素。
			\end{itemize}
		\end{column} \pause
		\begin{column}{0.5\textwidth}
			  如果单纯求第 $ k $ 大可以用可持久化线段树,修改怎么办?\pause
			
			  线段树套平衡树。
			
			  查询时二分, $ O(\log^3n) $ 。
		\end{column}
	\end{columns}
\end{qedframe}
\begin{qedframe}
	\frametitle{JSOI2008 火星人\hlink{http://www.lydsy.com/JudgeOnline/problem.php?id=1014}}
	\begin{columns}
		\begin{column}{0.5\textwidth}
			  给定长度为 $ n $ 的字符串,有 $ q $ 次操作,操作有三种:
			\begin{itemize}
				\item 修改单个字符;
				\item 插入单个字符;
				\item 查询两个位置开始的后缀的LCP。
			\end{itemize}
		\end{column} \pause
		\begin{column}{0.5\textwidth}
			  维护Hash值,二分判定。
			
			   $ O(\log^2n) $ 。
		\end{column}
	\end{columns}
\end{qedframe}
\begin{qedframe}
	\frametitle{HNOI2011 括号修复\hlink{http://www.lydsy.com/JudgeOnline/problem.php?id=2329}}
	给定长度为 $ n $ 的括号序列(仅含``\texttt{(}''和``\texttt{)}''),有 $ q $ 次操作,操作有四种:
	\begin{itemize}
		\item 区间赋值;
		\item 区间翻转;
		\item 区间反转;
		\item 给定区间,求至少要修改多少个位置的括号,才能使得区间中的序列变为合法括号序列。
	\end{itemize}
\end{qedframe}
\begin{qedframe}
	\frametitle{HNOI2011 括号修复 - 题解}
	重点在于维护,先考虑单次询问的情况。\pause
	
	先把序列中所有合法的部分(连着的``\texttt{()}'')全部消去,最后一定剩下一堆右括号接一堆左括号。此时的最优解一定是:对于右括号和左括号的两端,分别隔一个改一个:
	
	\begin{center}
	\texttt{)))(((((}
	
	\texttt{\underline{(})\underline{(}\underline{)}(\underline{)}(\underline{)}}
	\end{center}
	
	将之前删去的括号都插回去,不难证明仍为合法序列。因此这种方案为合法解。也可以证明这是最优解。
\end{qedframe}
\begin{qedframe}
	\frametitle{HNOI2011 括号修复 - 题解}
	扩展到静态多次询问问题呢?
	
	我们要维护什么?\pause
	\vskip 1em
	括号 $ \Longrightarrow $ \texttt{+1/-1}序列\pause
	\vskip 1em
	右括号的个数 $ \Leftrightarrow $ 左起的最小值;左括号的个数 $ \Leftrightarrow $ 右起的最大值。
	
	和最大子段和差不多。
\end{qedframe}
\begin{qedframe}
	\frametitle{HNOI2011 括号修复 - 题解}
	别的操作呢?
	
	翻转:\pause 没有影响。
	
	赋值:\pause 没有影响。
	
	反转:\pause 影响大了!
	
	\begin{itemize}
		\item 还需要维护左起最大值和右起最小值;
		\item 撞上了赋值标记怎么办?\pause 如果反转先来,打翻转和赋值标记;如果赋值先来,直接反转赋值标记;
		\item 三种操作的顺序如何?\pause 先翻转、再反转,最后赋值。
	\end{itemize}
\end{qedframe}

\subsection{Treap}
\begin{qedframe}
	\frametitle{Treap}
	另一种平衡树。
	
	Tree + Heap。
	
	在二叉树的基础上,存储随机的权重,并按照权重形成堆。
	
	可以证明期望树高 $ O(\log n) $ 。
	
	还可以证明修改的期望旋转次数为常数。
	
	因此可以方便地可持久化。(为什么Splay不好可持久化?)
\end{qedframe}
\begin{qedframe}
	\frametitle{Treap的操作}
	和二叉搜索树一样。
	
	操作完成用旋转操作后维护堆的性质。
\end{qedframe}
\begin{qedframe}
	\frametitle{无旋转Treap}
	基于两个操作:split和merge。
	
	merge类似左偏树的合并,不过不发生交换。
	
	split类似BST查找算法,不过会把找到的下边的分支接到上面。
	
	每次操作最多产生 $ O(\log n) $ 个节点,可以直接可持久化。而且可以任意切割合并,可以完成Splay的几乎所有操作。
\end{qedframe}

\subsection{其他平衡树}
\begin{qedframe}
	\frametitle{Size-Balanced Tree}
	国人提出的某种数据结构,通过维护某种子树的大小关系来保持平衡。
	
	似乎和AVL一般快,反正比Splay快。
	
	有兴趣的同学可以自学。
\end{qedframe}
\begin{qedframe}
	\frametitle{替罪羊树}
	Scapegoat Tree。
	
	非常暴力。
	
	定义了平衡因子 $ \alpha $ ,一般取 $ 0.6\sim 0.7 $ 。
	
	一旦发现有节点的较大儿子的大小超过了整棵子树的大小乘以 $ \alpha $ ,则将整棵子树重构为完全二叉树。如果有多个这样的节点,则选择深度最小的。
	
	可以证明均摊复杂度为 $ O(\log n) $ 。
	
	有兴趣的同学可以自学。
\end{qedframe}


\section[Part II]{Link-Cut Tree}
\title[数据结构及其应用]{\fontsize{18pt}{\baselineskip}\textbf{Link-Cut Tree}}
\subtitle{}
\maketitle

\subsection{基础知识}
\begin{qedframe}
	\frametitle{引入}
	Splay可以非常灵活地维护一个序列。
	
	树可以视为一条一条链。\pause
	\vskip 1em
	考虑有根树,以及一个点到根的链。
	
	按从根到这个点的顺序抽出节点的序列,用Splay维护。
	
	其他的点呢?\pause
	\vskip 1em
	序列左侧的元素是这个点的祖先,可能出现在右边的元素是这个点的子孙。
	
	我们知道Splay也是树。能不能把两棵树结合一下?
\end{qedframe}
\begin{qedframe}
	\frametitle{启发}
	我们需要Splay做什么?\pause
	
	——我们需要Splay维护上页中提到的序列,即点与点之间的辈分关系。\pause
	\vskip 1em
	Splay需要什么?\pause
	
	——Splay只需要这一偏序关系即可。\pause
	\vskip 1em
	Splay的操作对于我们需要维护的关系会产生怎样的影响?\pause
	
	——考虑简单情况:某个点有两个儿子。我们在这个点在Splay中的右儿子上接两个节点,再对其中一个儿子旋转。
	
	——即便是这么旋转之后,偏序关系也不会变。\pause
	\vskip 1em
	于是我们可以这么做——
\end{qedframe}
\begin{qedframe}
	\frametitle{定义}
	我们直接按照树的结构建Splay:树中节点的父亲就是自己在Splay中的父亲。
	
	但是初始时,我们让所有节点的左右儿子都为空。
	
	我们再定义\textbf{实边}为,父亲的儿子也有自己的点连向父亲的边。反之则是\textbf{虚边}。
	
	从Splay的角度来看,初始时相当于 $ n $ 条序列。
\end{qedframe}
\begin{qedframe}
	\frametitle{Expose操作}
	当我们需要一个点到根的路径时,我们需要把这些点在Splay中都串起来。
	
	也就是从这个点出发,一路向父亲走。如果发现某个点的父亲的右儿子不是自己,就改一下。
	
	当然,还得把最开始那个点的右儿子改为空。
	
	朴素的实现的话,最坏可能有 $ O(n) $ 。\pause
	
	于是Splay就登场啦。
	
	我们用splay操作将当前的点提到``实根'',然后直接改父亲就好。
	
	可以证明均摊复杂度为 $ O(\log n) $ 。
\end{qedframe}
\begin{qedframe}
	\frametitle{任意两点间的路径}
	如果我们不是要到根的路径呢?\pause
	
	只需要先expose一边,再对另一边做类似的操作。但是,我们需要在LCA的位置停住。
	
	因此,如果某次走到的父亲已经是Splay的根了,那么我们就找到LCA了。(是吗?)
	
	此时我们有两条链,拼起来就是整个路径了。
\end{qedframe}
\begin{qedframe}
	\frametitle{改变根节点}
	有时我们需要以另一个节点为根。不一定是题目要求,可能是别的操作的要求。\pause
	
	改变根节点,其实是颠倒了两个根之间的辈分关系。
	
	而其它点没有影响。\pause
	
	也就相当于翻转了这一条链。
	
	因此expose后打翻转标记即可。
\end{qedframe}
\begin{qedframe}
	\frametitle{加边}
	此时的图应当是森林,加边后仍然应当是森林。\pause
	
	两个连通块之间的辈分关系是无所谓的。因此其实就是让其中一个点成为另外整棵树的父亲。
	
	而且,边的另一个端点应当是整棵子树的根。\pause
	
	那么,只要让在一边换根之后,直接接起来就好。
\end{qedframe}
\begin{qedframe}
	\frametitle{删边}
	我们先假设这两点间一定有边。\pause
	
	首先要解决的问题是,怎么知道谁是父亲?\pause
	
	答案是不知道,只能两边都试一试。
	
	我们用类似加边的思考方法,先找到儿子那边的子树的根。\pause
	
	但是又不能找到父亲上面去。因此要先把父亲那边断开。
	
	于是先对父亲做expose的第一步:splay后断开右儿子。此时儿子的实树根的父亲应当是父亲了。
	
	不然我们就可以知道:一定是反着来的。
\end{qedframe}
\begin{qedframe}
	\frametitle{删边}
	如果两点间没边呢?\pause
	
	此时上面的方法就错了!仍然会断开一条边,从而将两点分成两个连通块。\pause
	
	那么我们换一种找法:expose了儿子之后,父亲一定是儿子的前驱。
	
	这个方法对于有边无边都是对的。
	
	当然要注意push。
\end{qedframe}
\begin{qedframe}
	\frametitle{判断连通性}
	我们可以用类似找两点间路径的方法。
	
	如果连通,那么在碰到Splay根后,此时根的实树中最右侧的节点一定是另一节点。
\end{qedframe}
\begin{qedframe}
	\frametitle{应用}
	用LCT可以干什么?\pause
	\begin{itemize}
		\item 链修改;
		\item 链查询;
		\item 改变树的形态。
	\end{itemize}
\end{qedframe}
\begin{qedframe}
	\frametitle{其他}
	能使用其他平衡树吗?\pause
	
	答案是可以,但是复杂度会变成 $ O(\log^2n) $ 。
	
	我也不知道为什么。有兴趣的同学可以看Tarjan的论文。
	
	对,这个也是Tarjan发明的。
\end{qedframe}

\subsection{例题}
\begin{qedframe}
	\frametitle{2012集训队互测 Tree\hlink{http://tsinsen.com/A1303}}
	\begin{columns}
		\begin{column}{0.5\textwidth}
			   $ n $ 个节点的树,每个点的初始点权为1。有 $ q $ 次操作,操作有四种:
			\begin{itemize}
				\item 将路径上的点权 $ +k $ ;
				\item 将路径上的点权 $ \times k $ ;
				\item 求路径上点权和;
				\item 删去一条边,加上另一条边,保证操作后仍为一棵树。
			\end{itemize}
		\end{column} \pause
		\begin{column}{0.5\textwidth}
			  最基础的应用。
			
			  说到底LCT也还是Splay,Splay维护起来和线段树也差不多。
			
			  要注意加和乘的标记应用顺序。
		\end{column}
	\end{columns}
\end{qedframe}
\begin{qedframe}
	\frametitle{HNOI2010 弹飞绵羊\hlink{http://www.lydsy.com/JudgeOnline/problem.php?id=2002}}
	\begin{columns}
		\begin{column}{0.5\textwidth}
			  地上有 $ n $ 个格子,每个格子上有数字 $ k_i $ 。
			
			  当绵羊到达第 $ i $ 个格子时,它会被弹到第 $ i+k_i $ 个格子,当 $ i+k_i>n $ 时,绵羊被弹飞。
			
			  有 $ q $ 次操作,操作有两种:
			\begin{itemize}
				\item 求绵羊从第 $ i $ 个格子出发,弹几次后会被弹飞;
				\item 修改一个格子的数字。
			\end{itemize}
		\end{column} \pause
		\begin{column}{0.5\textwidth}
			  不难发现,如果令每个格子指向其弹到的格子,弹飞的格子指向一个虚拟节点,那么就构成了一棵树。
			
			  用LCT维护这个树即可。
			
			  我们要知道的是链的长度。
			
			  也就是Splay左子树的大小。
			
			  有没有更简单的办法?\pause
			
			  Splay维护括号序列?
		\end{column}
	\end{columns}
\end{qedframe}
\begin{qedframe}
	\frametitle{WC2006 水管局长\hlink{http://www.lydsy.com/JudgeOnline/problem.php?id=2594}}
	\begin{columns}
		\begin{column}{0.5\textwidth}
			   $ n $ 个点 $ m $ 条边的图,边带权。有 $ q $ 次操作,操作有两种:
			\begin{itemize}
				\item 求两点间路径上最大边权的最小值;
				\item 删除一条边。
			\end{itemize}
		\end{column} \pause
		\begin{column}{0.5\textwidth}
			  一个结论是,能成为答案的边一定在最小生成树上。
			
			  (已经是常用结论了)
			
			  但是删边怎么办呢?\pause
			
			  倒过来,变成加边!\pause
			
			  加入一条边会得到一个环,删去环上的最大边。
		\end{column}
	\end{columns}
\end{qedframe}

\begin{qedframe}
	\frametitle{BZOJ3091 城市旅行\hlink{http://www.lydsy.com/JudgeOnline/problem.php?id=3091}}
	 $ n $ 个点的树,每个点有点权。有 $ q $ 次操作,操作有四种:
	\begin{itemize}
		\item 加入一条边;
		\item 删除一条边;
		\item 将路径上的点权 $ +k $ ;
		\item 询问在路径上任取两点,得到的子路径上的点权和的期望值。
	\end{itemize}
	
	你还需要判断给定的操作是否合法。每次操作后图应仍为森林。
\end{qedframe}
\begin{qedframe}
	\frametitle{BZOJ3091 城市旅行 - 题解}
	这个题最麻烦的地方在于维护期望值。
	
	我们先推一下式子:假设是一条长度为 $ n $ 的链,点权为 $ a_1,\ldots,a_n $ ,那么
	\begin{eqnarray*}
		S(l,r) & = & \sum_{i=l}^{r}{a_i} \\
		E & = & \pause\frac{2}{n(n+1)}\sum_{l=1}^{n-1}\sum_{r=l}^{n}{S(l,r)} \\
		  & = & \frac{2}{n(n+1)}\sum_{i=1}^{n}i(n-i+1)\cdot a_i \\ \pause
	\end{eqnarray*}
	如何在Splay中维护这个值?
\end{qedframe}
\begin{qedframe}
	\frametitle{BZOJ3091 城市旅行 - 题解}
	维护以下值:
	\begin{itemize}
		\item  $ size $ 为树的大小;
		\item  $ sum=\sum{a_i} $ ;
		\item  $ lm=\sum{i\cdot a_i} $ ;
		\item  $ rm=\sum{(size-i+1)\cdot a_i} $ ;
		\item  $ E $ 为期望。
	\end{itemize}
	当然,还有得增量。\pause
	
	另外,由于需要换根,在处理翻转标记的时候还需要交换 $ lm $ 和 $ rm $ 。
\end{qedframe}

\begin{qedframe}
	\frametitle{LNOI2014 LCA\hlink{http://www.lydsy.com/JudgeOnline/problem.php?id=3626}}
	 $ n $ 个节点的有根树,1号节点为根。
		
	有 $ q $ 次询问,每次需要求出编号在 $ [l,r] $ 间的节点各自与点 $ x $ 的LCA的深度之和,即:
	\[\sum_{i = l}^{r}{\mathrm{depth}[LCA(i,x)]}\]
\end{qedframe}
\begin{qedframe}
	\frametitle{LNOI2014 LCA - 题解}
	假设固定一个点,如何快速求出另外一个点与这个点的LCA深度?\pause
	
	和LCT有何关系?\pause
	
	如果固定多个点呢?\pause
	\vskip 1em
	将这些点到根的路径 $ +1 $ ,那么每个点到根的路径和就是要求的式子。\pause
	\vskip 1em
	区间怎么处理?\pause
	
	拆成两个前缀,离线处理。
\end{qedframe}

\subsection{CodeChef March Challenge 2014 - GERALD07}
\begin{qedframe}
	\frametitle{Chef and Graph Queries\hlink{https://www.codechef.com/MARCH14/problems/GERALD07}}
	给定一个 $ n $ 个点 $ m $ 条边的图,每条边编号为 $ 1\sim m $ 。有 $ q $ 次询问,每次给定 $ l $ 和 $ r $ ,询问当仅保留编号为 $ l\sim r $ 的边时,图中有多少个连通块。
	
	 $ n,m,q\leq 200000 $ 。可能有自环与重边。
\end{qedframe}
\begin{qedframe}
	\frametitle{Chef and Graph Queries - 题解}
	我们考虑按照 $ 1\sim m $ 的顺序依次向空图中加边。加入一条边时,只会有两种情况:
	\begin{enumerate}
		\item 合并了两个连通块;
		\item 形成了一个环。
	\end{enumerate} \pause
	
	先看第一种,只有合并两个连通块时才会产生贡献。我们要求的实际上就是一个区间中这类边有多少条。 \pause
	
	再看第二种,形成环则代表可以删去图中的一条边,并保持当前的连通性。
	
	不妨删去环上加入时间最早(即编号最小)的一条边。我们发现了什么?
\end{qedframe}
\begin{qedframe}
	\frametitle{Chef and Graph Queries - 题解}
	假设加入的边为 $ i $ ,删去的边为 $ a_i $ 。只有当 $ i $ 加入的时候才能``安全''删除 $ a_i $ ,否则会导致一个连通块分成两个。 \pause
	
	换句话说, $ i $ 会产生贡献 $ \iff $ 选择的区间中有 $ i $ 且没有 $ a_i $ 
\end{qedframe}
\begin{qedframe}
	\frametitle{Chef and Graph Queries - 题解}
	假设我们已求得 $ a $ 。当我们询问 $ [l,r] $ 时,我们实际上需要知道什么? \pause
	
	我们需要知道在 $ [l,r] $ 中有多少 $ a_i<l $ 。设其个数为 $ x $ ,连通块的个数就是 $ n-x $ 。 \pause
	
	用可持久化线段树可以轻松解决这一询问。
\end{qedframe}
\begin{qedframe}
	\frametitle{Chef and Graph Queries - 题解}
	至于求 $ a $ ,只需维护一棵LCT,按 $ 1\sim m $ 的顺序加边。
	
	加入一条边时,判断是否成环。如果构成了环,则找到环上编号最小的边的编号,即为 $ a_i $ 。
	
	如果没有成环,则记 $ a_i=0 $ 。注意当边为自环时, $ a_i=i $ 。
	
	总复杂度 $ O((n+m)\log n) $ 。
\end{qedframe}

\subsection{CodeChef November Challenge 2013 - MONOPLOY}
\begin{qedframe} % 21
	\frametitle{Gangsters of Treeland}
	给定一棵 $ n $ 个点的树, $ 1 $ 号节点为根。初始时每一个点都被染成了一种不同的颜色。如果一条边的两个端点颜色不同,则其费用为 $ 1 $ ,否则费用为 $ 0 $ 。
	
	有 $ q $ 次操作,操作有下面两种:
	\begin{itemize}
		\item 将从点 $ u $ 到根的路径上的所有点染成一种新的颜色。
		\item 询问点 $ u $ 子树中所有点走到根的费用的平均数。
	\end{itemize}
	
	 $ n,q\leq100000 $ 。
\end{qedframe}
\begin{qedframe}
	\frametitle{Gangsters of Treeland - Solution}
	直接维护每个点的颜色,查询时数链上有多少种不同的颜色? \pause
	
	带修改的树上第 $ k $ 大?或者是分块乱搞?无论哪个复杂度都太高。 \pause
	\vskip 1em
	为什么非得纠结颜色呢?可不可以利用``每次修改的颜色是一种新颜色''这样的性质?
\end{qedframe}
\begin{qedframe}
	\frametitle{Gangsters of Treeland - Solution}
	我们考虑直接维护边的费用。 \pause
	
	初始时所有边的费用均为 $ 1 $ ,修改一个点时,这个点到根的路径上的所有边的费用变为 $ 0 $ ,而其他和这条路径上的点相连的费用变为 $ 1 $ 。 \pause
	
	一次操作影响的边可能达到 $ O(n) $ ,直接操作是肯定不行的。一定还有别的性质。 \pause
	\vskip 1em
	每个点和儿子的连边中,至多一条费用为 $ 0 $ 。 \pause
	
	这似乎让人想起了什么……特别熟悉的东西…… \pause
	
	这不就是一棵Link-Cut Tree吗?
\end{qedframe}
\begin{qedframe}
	\frametitle{Gangsters of Treeland - Solution}
	费用为 $ 0 $ 的边就是LCT的实边,费用为 $ 1 $ 的边就是LCT的虚边。每次操作一个点就相当于expose(也称access)一个点。
	
	而一个点走到根的费用就是到根路径上的虚边条数。
	
	容易发现这个和原问题是等价的。 \pause
	\vskip 1em
	还记得LCT的复杂度吗?expose操作是均摊 $ O(\log n) $ 的。 \pause
	
	也就是说,expose时的``关键点'',即改变了实边的点,是均摊 $ O(\log n) $ 的。
	
	换言之,我们只有至多 $ O(n\log n) $ 次实质上的修改!
\end{qedframe}
\begin{qedframe}
	\frametitle{Gangsters of Treeland - Solution}
	我们实现一棵LCT,修改一个点时进行expose操作,每找到一个关键点就修改一次。
	
	而我们要维护的,就是每个点到根的虚边条数。 \pause
	
	这个就很简单了。求出DFS序之后建线段树,每次虚边和实边切换的时候就做一次段修改。查询则直接是段查询。
	
	总复杂度 $ O(n\log^2n) $ 。
\end{qedframe}
\begin{qedframe}
	\frametitle{Gangsters of Treeland - Extras}
	是不是觉得转化略神?
	
	实际上还可以更神。由于问题等价于LCT,我们可以增加其它LCT支持的操作,比如换根。 \pause
	
	回忆换根的实现:先把要提成根的点expose,再splay到根,之后再打翻转标记。那么我们可以给题目增加这么一个操作:换根,同时把新旧根之间的路径染成新的颜色。 \pause
	
	问题在于线段树那部分要怎么实现。我们需要支持子树查询、子树修改,以及换根。 \pause
	
	实际上也是可做的。这里就不说了,有兴趣的同学可以自己思考或者在课后与我交流。 \pause
	
	至于LCT的别的操作能不能支持呢?这个我没有细想,同样,有兴趣的同学可以自己思考这个问题。
\end{qedframe}


\section[Part III]{分块方法}
\title[数据结构及其应用]{\fontsize{18pt}{\baselineskip}\textbf{分块方法}}
\subtitle[]{SQRT-N Decomposition}
\maketitle

\subsection{分块方法}
\begin{qedframe}
	\frametitle{朴素分块方法}
	直接将序列分成 $ O\left(\sqrt{n}\right) $ 块。\pause
	
	每个区间会对应 $ O\left(n/\sqrt{n}\right)=O\left(\sqrt{n}\right) $ 块,以及 $ O\left(\sqrt{n}\right) $ 个单独的元素。
	
	因此单次操作的复杂度为 $ O\left(\sqrt{n}\right) $ 。配合标记、重构,可以完成大多数操作。\pause
	
	根据不同操作的复杂度、出现频率,可以调整块的大小,达到更优的复杂度。
\end{qedframe}
\begin{qedframe}
	\frametitle{按大小分类}
	有时候我们不一定非要把序列分块:我们可以按照权值、出现次数、参数大小等分类。
	
	说起来比较抽象,通过例题体会一下。
\end{qedframe}
\begin{qedframe}
	\frametitle{CF80D Time to Raid Cowavans\hlink{http://codeforces.com/contest/103/problem/D}}
	\begin{columns}
		\begin{column}{0.5\textwidth}
			  给定长度为 $ n $ 的序列 $ a[1\sim n] $ ,有 $ q $ 次询问。
			
			  每次询问给定 $ (x,y) $ ,求 $ a[x]+a[x+y]+a[x+2y]+\cdots $ 
		\end{column} \pause
		\begin{column}{0.5\textwidth}
			  要求的东西不连续,传统的数据结构无用武之地。
			
			   $ y $ 的取值任意,也不好预处理。\pause
			\vskip 1em
			  不妨按照 $ y $ 的大小分类:\pause
			\begin{itemize}
				\item 如果 $ y\leq\sqrt{n} $ ,那么预处理出每个位置的答案,用类似后缀求和的方法可以做到 $ O\left(n\sqrt{n}\right) $ ;
				\item 如果 $ y>\sqrt{n} $ ,那么一次询问设计的元素个数不超过 $ n/\sqrt{n}=O\left(\sqrt{n}\right) $ ,直接统计。
			\end{itemize}
		\end{column}
	\end{columns}
\end{qedframe}
\begin{qedframe}
	\frametitle{经典题目}
	\begin{columns}
		\begin{column}{0.5\textwidth}
			  给定 $ n $ 个点 $ m $ 条边的无向图,每个点是黑色或者白色。有 $ q $ 次操作,操作有两种:
			\begin{itemize}
				\item 将一个点反色;
				\item 询问与一个点直接相连的黑色点的个数。
			\end{itemize}
		\end{column} \pause
		\begin{column}{0.5\textwidth}
			  直观感受可以得出,绝大多数点的度数都比较小。
			
			  但是度数大的点还是可能存在不少。\pause
			
			  因此按照度数分类:
			\begin{itemize}
				\item 对于度数 $ \leq\sqrt{n} $ 的点,修改/询问时枚举相连的点;
				\item 对于度数 $ >\sqrt{n} $ 的点,这类点的个数最多 $ O\left(m/\sqrt{n}\right)=O\left(\sqrt{n}\right) $ 个,因此在修改其他点时,同时更新这些点的答案;修改这类点时不管其它点,询问时直接输出答案。
			\end{itemize}
		\end{column}
	\end{columns}
\end{qedframe}
\subsection{莫队算法}
\begin{qedframe}
	\frametitle{莫队算法}
	莫队算法是处理区间问题的离线算法,相当万能。
	
	它的思路是:我们将询问区间以某种顺序重排,然后依次处理。从前一个区间移到下一个时,\textbf{暴力移动}:一个一个删除、一个一个插入。
	
	这么暴力的方法,可以做到怎样的复杂度?\pause
	\vskip 1em
	答案是 $ O\left(n\sqrt{n}\right) $ 次插入/删除。
	
	将区间按照左端点分块,同一块内的按照右端点排序。
\end{qedframe}
\begin{qedframe}
	\frametitle{Tsinsen A1206 小Z的袜子\hlink{http://tsinsen.com/A1206}}
	\begin{columns}
		\begin{column}{0.5\textwidth}
			  长度为 $ n $ 的序列,有 $ q $ 个询问。
	
			  每次给定区间 $ [l,r] $ ,问在区间中随机抽取两个元素,其值相同的概率。
		\end{column} \pause
		\begin{column}{0.5\textwidth}
			  传统的数据结构几乎无从下手。
			
			  但是使用莫队算法,可以轻易做到 $ O\left(n\sqrt{n}\right) $ 。
		\end{column}
	\end{columns}
\end{qedframe}
\subsection{2014 ACM/ICPC Asia Regional Xi'an Online - C}
\begin{qedframe}
	\frametitle{Paint Pearls}
	给定一个长度为 $ n $ 的序列,每一位有一个目标颜色。初始时每一位都没有颜色。每次可以选择一个区间,将区间内的所有元素改为其\textbf{目标颜色}。设区间内不同颜色的数量为 $ x $ ,则操作的代价为 $ x^2 $ 。求最小代价。
	
	 $ n\leq 50000 $ 。
\end{qedframe}
\begin{qedframe}
	\frametitle{Paint Pearls - Solution}
	朴素的DP为:令 $ f[i] $ 表示前 $ i $ 个元素变为目标颜色的最小代价,方程为
	\[f[i]=\min\left\{f[j-1]+w(j,i)\right\}\]
	复杂度 $ O(n^2) $ 。
\end{qedframe}
\begin{qedframe}
	\frametitle{Paint Pearls - Solution}
	容易发现答案的上界为 $ n $ ---每次操作一个元素即可。 \pause
	
	基于上界又可以发现,如果一个区间内有超过 $ \sqrt{n} $ 种颜色,我们一定不会去操作它。 \pause
	
	换句话说,对于一个 $ f[i] $ ,只需要考虑 $ \sqrt{n} $ 个不同的 $ w(j,i) $ 的取值。 \pause
	
	而 $ f $ 显然单调不减,因此 $ w(j,i) $ 相同时选择最靠左的 $ f[j-1] $ 。 \pause
	
	我们只需知道这些 $ j $ 的值。
\end{qedframe}
\begin{qedframe}
	\frametitle{Paint Pearls - Solution}
	只要记录当前位置往左出现的前 $ \sqrt{n} $ 种颜色,及对应位置即可。
	
	移动到下一个数时,如果颜色出现在了前 $ \sqrt{n} $ 种之中,则暴力删除并移到最前面。
	
	总复杂度 $ O\left(n\sqrt{n}\right) $ 。 \pause
	\vskip 1em
	其实,数据中最优解选的每个区间都只有不超过5种颜色……
\end{qedframe}
\subsection{CodeChef June Challenge 2014 - SEAARC}
\begin{qedframe}
	\frametitle{Sereja and Arcs}
	给定一个长度为 $ n $ 的序列 $ a $ ,对于任意的 $ x\neq y $ ,如果 $ a[x]=a[y] $ ,则在 $ x $ 和 $ y $ 之间画一条弧。称两条弧 $ x,y $ 和 $ l,r $ 相交当且仅当 $ x<l<y<r $ 或者 $ l<x<r<y $ 。求有多少条异色弧相交。
	
	 $ n\leq 100000 $ , $ a[i]\leq 100000 $ 。时限5s。
\end{qedframe}
\begin{qedframe}
	\frametitle{Sereja and Arcs - 题解}
	我们把圆弧视为线段,问题就是求有多少对有交且端点形如\texttt{1212}的异色线段。 \pause
	\vskip 1em
	将颜色按出现次数分类,次数大于 $ T $ 的为大块,其他的为小块。
	
	那么我们需要处理三种情况:
	\begin{enumerate}
		\item 大块之间的贡献;
		\item 小块与大块之间的贡献;
		\item 小块之间的贡献。
	\end{enumerate}
\end{qedframe}
\begin{qedframe}
	\frametitle{Sereja and Arcs - 题解}
	\textbf{大块之间的贡献} \pause

	我们枚举\texttt{1212}中的第二个\texttt{1},再枚举另外一种颜色。
	
	可以发现,答案为第二个\texttt{1}右边\texttt{2}的出现次数,乘上左边每个\texttt{2}的左侧的\texttt{1}的个数之和。 \pause
	
	由于大块个数不超过 $ O\left(\dfrac{n}{T}\right) $ ,我们可以预处理每个位置左侧每种颜色的出现次数。
	
	用这个就能维护,对于某种颜色 $ col $ ,和某个位置以左与这个位置颜色相同的所有位置,其左侧的 $ col $ 颜色的个数之和。
	
	计算时枚举一个位置和一种颜色。总复杂度 $ O\left(\dfrac{n^2}{T}\right) $ 。
\end{qedframe}
\begin{qedframe}
	\frametitle{Sereja and Arcs - 题解}
	\textbf{小块与大块之间的贡献} \pause

	枚举每个大块,再枚举每个小块。 \pause
	
	按照小块的每个元素把大块的元素划成若干区间,求出每个区间里的大块元素个数,记这个序列为 $ a $ 。
	
	假设我们枚举小块的左右端点,那么大块的两个元素的选择就是
	\begin{center}
		\texttt{(左边+右边)×中间}
	\end{center}
\end{qedframe}
\begin{qedframe}
	\frametitle{Sereja and Arcs - 题解}
	\textbf{小块与大块之间的贡献} \pause
	
	考虑当 $ a $ 的某一个元素成为``中间''部分时的贡献,我们需要求出这个元素为``中间''时可能的``左右''之和。
	
	这个元素左边第一块会被算一次,第二块被算两次,以此类推。整个这一部分还要再乘上右端点的可能的个数。
	
	右边也是类似的。这一部分是可以通过一些简单的预处理求出来的。
	
	那么复杂度也是 $ O\left(\dfrac{n^2}{T}\right) $ 。
\end{qedframe}
\begin{qedframe}
	\frametitle{Sereja and Arcs - 题解}
	\textbf{小块之间的贡献} \pause

	先考虑一个 $ O(n^3) $ 的算法,枚举\texttt{1212}中\texttt{2}的左右端点,再枚举另外一种颜色。
	
	此时相交的线段数就是该颜色在两个\texttt{2}之间的出现次数,乘上第一个\texttt{2}左边的出现次数。 \pause
	
	如果左端点从右往左扫,则可以在 $ O(1) $ 的时间内处理变化,并直接计算对答案的贡献的增量,从而优化到 $ O(n^2) $ 。 \pause
	
	令 $ f[i] $ 代表在当前右端点下,左端点为 $ i $ 时相交的线段数。不妨考虑,右端点右移时, $ f $ 会如何变化。
	
	如果我们能维护 $ f $ ,那么只需要枚举与右端点颜色相同的位置。
\end{qedframe}
\begin{qedframe}
	\frametitle{Sereja and Arcs - 题解}
	\textbf{小块之间的贡献} \pause
	
	我们考虑只枚举与右端点颜色相同的位置,此时对 $ f $ 的影响是按照这些位置分段的,每一段内的增量相同。
	
	也就是说,右端点右移会导致若干个段的修改,因此我们可以用树状数组维护答案。 \pause
	
	令 $ x_i $ 为颜色 $ i $ 的出现次数,复杂度应该是 $ O\left(\left(\sum x^2_i\right)\log n\right) $ 。
	
	而 $ \max\left\{\sum x^2_i\right\} $ 其实是 $ O(nT) $ 级别的。
	
	考虑给某个 $ x_i $ 加 $ 1 $ 带来的增量 $ \Delta=2x_i+1 $ ,因此一定会选择最大的 $ x_i $ 去加1。
	
	换句话说,当有 $ \frac{n}{T} $ 个 $ x_i $ 为 $ T $ 时原式取到最大值。因此复杂度为 $ O(nT\log n) $ 。
\end{qedframe}
\begin{qedframe}
	\frametitle{Sereja and Arcs - 题解}
	综上,取
	\[T=\sqrt{\frac{n}{\log n}}\]
	可得最优复杂度 $ O\left(n\sqrt{n\log n}\right) $ 。时限有5s,毫无压力。 \pause
	
	但可以发现, $ O\left(\dfrac{n^2}{T}\right) $ 的有两个部分, $ O(nT\log n) $ 的只有一部分,因此应适当调大 $ T $ 。
	
	经过实测,取 $ T=\left\lfloor1.8\cdot\sqrt{\frac{n}{\log n}}\right\rfloor $ 时效果最好,只需要2.48s即可通过全部数据。
\end{qedframe}


\section{结束语}
\begin{qedframe}
	\frametitle{在结束之前}
	如何学好数据结构?
	\vskip 1em
	如何在考试中做出数据结构题?
	\vskip 1em
	真的能在考试中写对数据结构吗?
	\vskip 1em
	NOI水平的数据结构题大概是什么样的难度?
\end{qedframe}

\appendix
\section{}
\begin{qedframe}
	\frametitle{Fin.}
	\centering
	\LARGE
	谢谢大家!欢迎课后交流。
\end{qedframe}

\end{document}
